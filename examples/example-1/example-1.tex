
% ##########################################################################################################
\documentclass[a5paper]{article}
\usepackage{lipsum}
\usepackage[columnwise]{lineno}
\usepackage{coffeexelatex}
\usepackage[top=10mm]{geometry}
% \usepackage[headheight=10mm]{geometry}
\usepackage{leading}
\usepackage{parskip}
% \usepackage{etoolbox}
% \usepackage{/Volumes/Storage/cnd/node_modules/jizura-xelatex/tex-inputs/background-grid}
\usepackage{xeCJK}
\setCJKmainfont{Sun-ExtA}

\newenvironment{jzrplain}{%
% ..........................................................................................................
% begin
  % thx to http://www.tug.org/TUGboat/tb28-1/tb88bazargan.pdf for the next two settings:
  \lineskiplimit=-10pt%
  \lineskip=0pt%
  \topskip=0pt%
  % \setlength{\parskip}{4mm}%
  \setlength{\parindent}{0mm}%
  % \fontsize{\nominalfontsize}{\gridstrutlength}%
  \leading{4mm}
  % \makeatletter
  % \preto{\@verbatim}{\topsep=0pt \partopsep=0pt }
  % \makeatother
  }%
% end
  {\par}


\renewcommand{\CXmainroute}{../../lib/main}
\renewcommand{\CXtempoutroute}{/tmp/CXtempout.tex}
\renewcommand{\CXtemperrroute}{/tmp/CXtemperr.tex}
\renewcommand{\CXhost}{localhost:8910}
% \newcommand{\noderun}[2]{\noderunscript{\CXmainroute}{#1}{#2}}

% ##########################################################################################################
\begin{document}
\begin{jzrplain}
% \linenumbers

\CX

\clearpage%\mbox{}\clearpage

\section{Introduction}\label{intro}
% \lipsum[1]

(1) The original technique to execute an arbitrary command:
\begin{verbatim}
\immediate\write18{node
  "\CXmainroute"
  "\currfilepath"
  "helo"
  "readers (one)"
  > /tmp/temp.dat}\input{/tmp/temp.dat}
\end{verbatim}

(2) With ugly details largely hidden, the \verb#\exec{}# command is still fully general:
\begin{verbatim}
\exec{node
  "\CXmainroute"
  "\currfilepath"
  "helo"
  "readers (two)"}
\end{verbatim}

(3) \verb#\noderunscript{}# will execute NodeJS code that adheres to the call convention established by
\CX:
\begin{verbatim}
\noderunscript
  {\CXmainroute}
  {\currfilepath}
  {helo}
  {readers (three)}
\end{verbatim}

(4) Like the previous example, but with standard values assumed as shown above. This is the form that you
will want to use most of the time:
\begin{verbatim}
\noderun{helo}{readers (four)}
\end{verbatim}


{\textbf{Outputs:}}

\immediate\write18{node "\CXmainroute" "\currfilepath" "helo" "readers (one)" > /tmp/temp.dat}\input{/tmp/temp.dat}

\exec{node "\CXmainroute" "\currfilepath" "helo" "readers (two)"}

\noderunscript{\CXmainroute}{\currfilepath}{helo}{readers (three)}

\noderun{helo}{readers (four)}

\clearpage
\section{Configuration}\label{config}

To use \CX, put
\begin{verbatim}
\usepackage{coffeexelatex}
\end{verbatim}
into the header section of your \LaTeX\ file.

You may also want to include these lines in your \LaTeX\ document; these define, respectively, the route
to the \CX\ executable (\verb#coffeexelatex/lib/main.js#) and the temporary file that is used to communicate
between \TeX\ and your scripts (relative routes are resolved with respect to the current working directory,
so if you set a relative route, you must always run \TeX\ from within the same directory):
\begin{verbatim}
\renewcommand{\CXmainroute}{../../lib/main}
\renewcommand{\CXtempoutroute}{/tmp/coffeexelatex.tex}
\end{verbatim}


\clearpage
\section{Geometry}\label{geo}

% Write arbitrary text into the aux file:
\auxc{this line goes to the aux file}

% Write geometry data into the aux file:
\auxgeo

Use geometry data from aux file to render a table of layout dimensions into the document;
note the we could have used the \verb#\auxgeo# command anywhere in the document and that
this currently only works for documents with a single, constant layout.

Also note we're using a dash instead of an underscore here—in \TeX, underscores are special, so
we conveniently allow dashes to make things easier. The \CX\ command \verb#show-geometry# does
not take arguments, which is why the second pair of braces has been left empty:

{\textbf{Command:}}
\begin{verbatim}
\noderun{show-geometry}{}
\end{verbatim}

{\textbf{Output:}}

\noderun{show-geometry}{}

\clearpage
\section{Character Escaping}\label{esc}
The \CX\ command \verb#show-special-chrs# demonstrates that it is easy to include \TeX\ special characters
in the return value. The simple rule is that whenever the output of a command is meant to be understood
literally, it should be \verb#@escape#d:

{\textbf{Command:}}

\begin{verbatim}
\noderun{show-special-chrs}{}
\end{verbatim}

{\textbf{Output:}}

\noderun{show-special-chrs}{}

\clearpage
\section{The \texttt{aux} Object}\label{aux}



\subsection{Labels}\label{labels}

\CX\ will try and collect all labels from the \verb#*.aux# file associated with the current job; from
inside your scripts X X X X X X X X X X X X X X X X X X X X X X X X X X X X X X X X X X X X X X X X

{\textbf{Command:}}

\begin{verbatim}
\noderun{show-aux}{}
\end{verbatim}

{\textbf{Output:}}

{\fontsize{3mm}{3mm}\noderun{show-aux}{}}

\subsection{Evaluating Expressions}\label{evalcs}

The commands \verb#\evalcs{}# and \verb#\evaljs{}# allow you to evaluate an arbitrary self-contained
expression, written either in CoffeeScript or in JavaScript:

{\textbf{Command:}}

\begin{verbatim}
$23 + 65 * 123 = \evalcs{23 + 65 * 123}$
\end{verbatim}

{\textbf{Output:}}

$23 + 65 * 123 = \evalcs{23 + 65 * 123}$


\clearpage
\section{Curl}\label{curl}

% \newcounter{char}
% \setcounter{char}{1}

% \loop\ifnum\value{char}<27
% \edef\c{\Alph{char}}
% \expandafter\expandafter\expandafter\expandafter\expandafter\expandafter\expandafter\def\expandafter\expandafter\expandafter\csname\expandafter\expandafter\expandafter b\expandafter\c\expandafter\endcsname\expandafter{\expandafter\mathbb\expandafter{\c}}
% \addtocounter{char}{1}
% \repeat



\urlescape{abcd/efgh\%foo\#bar baz}

\urlescape{Li YongQiang}


\newcommand{\curl}[4]{%
  \execdebug{curl --silent --show-error #1 #2/#3#4}
  }

% \execdebug{curl -s 127.0.0.1:8910/foobar.tex/helo/\urlescape{Li YongQiang}}

% \newfontfamily\cjkIdeographFont{Sun-ExtA}

\execdebug{curl --silent --show-error 127.0.0.1:8910/foobar.tex/helo/黎永強}

\curl{}{127.0.0.1:8911}{foobar.tex/helo/}{黎永強}

% \CXdebug
% \verbatiminput{\CXtemperrroute}

% \CXiffileempty{\CXtemperrroute}%
%   {\CXiffileempty{\CXtempoutroute}%
%     {}%
%     {\verbatiminput{\CXtempoutroute}}}%
%   {{\color{red}\verbatiminput{\CXtemperrroute}}}

\CXtemperrroute\ is \CXiffileempty{\CXtemperrroute}{indeed}{not} empty.
\CXtempoutroute\ is \CXiffileempty{\CXtempoutroute}{indeed}{not} empty.


% \CXdebug

% \newread\reader

% \def\eolmarker{\par}

% \def\testifempty#1{%
% \openin\reader=#1\relax
% \ifeof\reader
%     \message{^^J#1: doesn't exist^^J}%
% \else
%     \read\reader to \readmacro
%     \ifeof\reader
%         \ifx\readmacro\eolmarker
%             \message{^^J#1: is empty^^J}%
%         \else
%             \message{^^J#1: is not empty^^J}%
%         \fi
%     \else
%         \message{^^J#1: is not empty^^J}%
%     \fi
% \fi
% \closein\reader
% }

% currfiledir: \currfiledir

% currfilebase: \currfilebase

% currfileext: \currfileext

% currfilename: \currfilename

% currfilepath: \currfilepath

% \jobname

% ##########################################################################################################
\end{jzrplain}
\end{document}
